\section{Week 4: 2nd - 8th February 2026}

\subsection{Thursday 5th February}

\subsubsection{Meeting 4: Zhaoting and Alkistis}

Since our last meeting I mainly just reviewed the code that I had written and tried to connect what I did there to the theory I read in the papers.

\begin{itemize}
    \item Went over the significance of why removing low values of k para affects the data so much.
    \begin{itemize}
        \item Small k para relates to the very high frequency values. This relates to the smooth foreground we want to remove.
        \item The small eigenvalues of the signal relate to the modes. The smallest modes correspond to the foreground.
    \end{itemize}
\item If we add polarised foregrounds to the smooth foregrounds, if we remove the small modes, we will remove the smooth foregrounds but also some of the polarised foregrounds.
\item The more k para and modes we remove, the closer our data will look to the signal since our HI signal has very large modes and is very oscillatory. However, that mean we will be increasing the amount of signal we are removing.
\item We want to make sure to balance the amount of foreground and noise we are removing with the amount of signal we are keeping. 
\item With PCA, we are just manually looking at which modes and k values correspond to what frequencies. With GPR, we will be fitting kernels to as closely mimic our foreground and noise and signal. We mainly just want to closely match the covariance.
\item Our next meeting will be on Tuesday at 11am on Zoom only with Zhaoting since Alkistis is not available.
\begin{itemize}
    \item Zhaoting will send me notebook 3 with the next steps of the project and will start on GPR and fitting different kernels to try to match our foregrounds and signals. He will send me this on Monday.
    \item Over the weekend, I should try to extract the eigenvectors instead of eigenvalues. This will help me better visualise and understand how the eigenvalues/eigenvectors connect with the frequencies and modes, etc. I should also try to plot the cylindrical power spectrum with the foreground signal or noise (which is generated by using np.gaussian.random I think). This is just to better understand how the cylindrical power spectrum relates to the frequencies and signals.
\end{itemize}
\end{itemize}

To do before Monday:

\begin{itemize}
    \item From notebook 2:
    \begin{itemize}
        \item Extract the eigenvectors and try to understand what it shows and why it's significant.
        \item Plot the cylindrical power spectrum for the foreground noise and gaussian noise and try to understand what it shows and why it's significant.
    \end{itemize}
\end{itemize}